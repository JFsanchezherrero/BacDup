\pagenumbering{roman} 
\setcounter{page}{1} 
\pagestyle{plain}

\chapter{Requerimientos y código}\label{apA}

Para la realización de este TFM se ha trabajado íntegramente en un entorno Linux por lo que los comandos de instalación utilizados han sido en este lenguaje desde la consola.

\section{$python$}

Los módulos instalados en el ambiente $python$ generado específicamente para este TFM se pueden encontrar en la siguiente url al repositorio de github *****


\textbf{Paquetes $pip$}: Los siguientes módulos y respectivas versiones fueron instalados para el correcto funcionamiento del programa desarrollado con el siguiente comando: 

\vspace{3mm}
\colorbox{gray}{\textcolor{white} {\$ pip install nombre\_programa}}
\vspace{3mm}

\begin{itemize}
    \item bcbio-gff 0.6.6: Librería que permite leer archivos GFF
    \item biopython 1.78: Conjunto de aplicaciones y programas con aplicaciones bioinformáticas.
    \item ftputil 4.0.0: Librería que permite acceder a servidores FTP.
    \item numpy 1.19.2: Paquete que permite crear vectores y matrices multidimensionales grandes y provee de funciones matemáticas de alto nivel para trabajar con ellas.
    \item pandas 1.1.2: Biblioteca que en combinación con NumPy permite manipular los datos y crear tablas o series temporales.
    \item wget 3.2: herramienta que permite la descarga de contenidos desde servidores web.
\end{itemize}

\textbf{BLAST}: \colorbox{gray}{\textcolor{white} {\$ sudo apt-get install ncbi-blast+}}
Con este comando se descarga la última versión de BLAST, si se prefiere descargar otra versión diferente, NCBI dispone de una carpeta ftp desde la que se puede descargar una versión específica: \url{ftp://ftp.ncbi.nlm.nih.gov/blast/executables/blast+/}
Una vez instalado, se puede utilizar directamente desde la terminal. Para generar una base de datos, utilizaremos los siguientes comandos:

\vspace{3mm}
\colorbox{gray}{\textcolor{white} {\$ makeblastdb -in fasta\_file -input\_type fasta -dbtype prot -out name}}
\vspace{3mm}

Donde \textit{fasta\_file} es el fichero de proteínas; \textit{name} es el nombre que le daremos a la nueva base de datos; \textit{input\_type} es el formato del fichero proporcionado; y \textit{dbtype} es el tipo de secuencias proporcionadas (en nuestro caso, proteínas).

Una vez generada la base de datos, se podrá hacer la búsqueda de alineamientos con blastp con el siguiente comando:

\vspace{3mm}
\colorbox{gray}{\textcolor{white} {\$ blastp -query fasta\_file -db name -outfmt '6 std qlen slen' -num\_threads X -out name\_out}}
\vspace{3mm}

Donde \textit{query} son las secuencias problema; \textit{num\_threads} es el número de CPUs que se utilizarán (una, en nuestro caso); y \textit{outfmt} es el tipo de formato que queremos darle a nuestra tabla de salida. En este caso, se elige el formato 6 con las columnas estándar y además se añaden las columnas de longitud de las secuencias problema y de referencia. Se puede encontrar más información sobre el formato 6 en \cite{scholz_blastn_nodate}

Todos los scripts de python para este TFM se pueden encontrar en la siguiente url de github: **************

\newpage
\section{$\mathbb{R}$}

A continuación se muestra la información de la sesión de R utilizada para la generación de gráficos circulares con los resultados de python:

\begin{verbatim}
> sessionInfo()
R version 4.0.3 (2020-10-10)
Platform: x86_64-pc-linux-gnu (64-bit)
Running under: Ubuntu 20.04.1 LTS

Matrix products: default
BLAS:   /usr/lib/x86_64-linux-gnu/blas/libblas.so.3.9.0
LAPACK: /usr/lib/x86_64-linux-gnu/lapack/liblapack.so.3.9.0

locale:
 [1] LC_CTYPE=es_ES.UTF-8       LC_NUMERIC=C               LC_TIME=es_ES.UTF-8       
 [4] LC_COLLATE=es_ES.UTF-8     LC_MONETARY=es_ES.UTF-8    LC_MESSAGES=es_ES.UTF-8   
 [7] LC_PAPER=es_ES.UTF-8       LC_NAME=C                  LC_ADDRESS=C              
[10] LC_TELEPHONE=C             LC_MEASUREMENT=es_ES.UTF-8 LC_IDENTIFICATION=C       

attached base packages:
[1] stats     graphics  grDevices utils     datasets  methods   base     

other attached packages:
[1] dplyr_1.0.2      colorspace_2.0-0 BioCircos_0.3.4 

loaded via a namespace (and not attached):
 [1] Rcpp_1.0.5         magrittr_2.0.1     tidyselect_1.1.0   R6_2.5.0          
 [5] rlang_0.4.9        fansi_0.4.1        plyr_1.8.6         tools_4.0.3       
 [9] cli_2.2.0          htmltools_0.5.0    ellipsis_0.3.1     yaml_2.2.1        
[13] digest_0.6.27      assertthat_0.2.1   tibble_3.0.4       lifecycle_0.2.0   
[17] crayon_1.3.4       purrr_0.3.4        RColorBrewer_1.1-2 htmlwidgets_1.5.3 
[21] vctrs_0.3.6        glue_1.4.2         compiler_4.0.3     pillar_1.4.7      
[25] generics_0.1.0     jsonlite_1.7.2     pkgconfig_2.0.3 \end{verbatim}


El script completo para la creación de BioCircos con los datos generados en dup\_annot.py se puede encontrar en el siguiente enlace: *********************